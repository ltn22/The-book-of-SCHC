\chapter{Getting started with openSCHC}

OpenSCHC provides for an easy way to understand and experiment with both the compression/decompression mechanism and the fragmentation protocol of SCHC. We start by illustrating the compression and fragmentation processes as defined in \rfc{8724}. To keep things simple, openSCHC does not have to run on an actual LPWAN network, experiments can be run on a regular network using UDP tunnels.


\section{Installation}\label{chap-plat}

\subsection{Downloading the code}

In order to download openSCHC, just clone the GitHub openschc repository onto your computer,


\begin{termc}[backgroundcolor=\color{palerod}, basicstyle=\ttfamily\small]
> git clone git@github.com:openschc/openschc.git
\end{termc}

and select the "scapy" branch.

\begin{termc}[backgroundcolor=\color{palerod}, basicstyle=\ttfamily\small, escapechar=@]
> git checkout scapy
\end{termc}

\subsection{Getting IPv6 connectivity}

We want to try out the openSCHC compression and fragmentation features on IPv6 traffic (as illustrated in \rfc{8724}). We therefore need to make our machine able to receive IPv6 traffic. Furthermore, we would like our machine to be able to receive IPv6 traffic destined to several IPv6 hosts (see \ref{chap-TestNetTopo}).

~

A simple way to get a global IPv6 prefix is to use the services of a tunnel broker such as \Index{Hurricane Electric}, see Figure~\ref{fig-IPv6conn}.

 \begin{figure}[!ht] 
\centering 


	\begin{tikzpicture}[scale=0.8, transform shape]

	\draw (0,0) node(SCHC)[rectangle, minimum size=2cm, draw, fill=palerod]{}; 
	\draw(SCHC) node {\small{SCHC}};
	

	\path (SCHC) -- ++(5, 3) coordinate (HE);
	
	\draw (HE) node(HEBox) [cloud, minimum width=3cm, minimum height= 2cm, draw, fill=pink!20] {};
	\draw (HE) node  {\small{HE}};

	\draw [double, dashed] (HEBox.south) to [bend left=45] node [above, sloped] {IPv6} node [below, sloped] {over IPv4 tunnel} (SCHC.east);
	

	\end{tikzpicture}
\caption{setting up IPv6 connectivity} 
\label{fig-IPv6conn} 
\end{figure} 

To set up the IPv6 tunnel:

\begin{itemize}
\item go to the website \url{https://tunnelbroker.net/}, 
\item create an account if needed or just log in, 
\item select \textit{Create another Tunnel},
\item enter the IPv4 address of the machine running openSCHC and a location,
\end{itemize}

Once the tunnel is created, you will find some configuration examples, which will help you  configuring your machine.

For Ubuntu 20, it will be in the \texttt{/etc/\Index{netplan}} directory:

\begin{termc}[backgroundcolor=\color{palerod}, basicstyle=\ttfamily\tiny, escapechar=@]
> cat /etc/netplan/50-cloud-init.yaml
# This file is generated from information provided by
# the datasource.  Changes to it will not persist across an instance.
# To disable cloud-init's network configuration capabilities, write a file
# /etc/cloud/cloud.cfg.d/99-disable-network-config.cfg with the following:
# network: {config: disabled}
network:
    ethernets:
        ens3:
            dhcp4: true
            match:
                macaddress: fa:16:3e:e9:db:5d
            addresses:
            - "2001:41d0:404:200:0:0:0:3a86/64"
            gateway6: "2001:41d0:0404:0200:0000:0000:0000:0001"
            set-name: ens3
    version: 2
    @\underline{tunnels:}@
      @\underline{he-ipv6:}@
       @\underline{mode: sit}@
       @\underline{remote: 216.66.87.102}@
       @\underline{local: 51.91.121.182}@
       @\underline{addresses:}@
         @\underline{- "2001:470:1f20:1d2::2/64"}@
       @\underline{gateway6: "2001:470:1f20:1d2::1"}@
\end{termc}

\noindent 
Type the following command to validate your new network configuration:
\begin{termc}[backgroundcolor=\color{palerod}, basicstyle=\ttfamily\tiny, escapechar=@]
>sudo netplan try
Warning: Stopping systemd-networkd.service, but it can still be activated by:
  systemd-networkd.socket
Do you want to keep these settings?


Press ENTER before the timeout to accept the new configuration


Changes will revert in 111 seconds
Configuration accepted.
>sudo netplan apply
\end{termc}
and your machine interfaces should now look this:



\begin{termc}[backgroundcolor=\color{palerod}, basicstyle=\ttfamily\tiny, escapechar=@]
> ifconfig
ens3: flags=4163<UP,BROADCAST,RUNNING,MULTICAST>  mtu 1500
        inet 51.91.121.182  netmask 255.255.255.255  broadcast 0.0.0.0
        inet6 2001:41d0:404:200::3a86  prefixlen 64  scopeid 0x0<global>
        inet6 fe80::f816:3eff:fee9:db5d  prefixlen 64  scopeid 0x20<link>
        ether fa:16:3e:e9:db:5d  txqueuelen 1000  (Ethernet)
        RX packets 9543878  bytes 1758163442 (1.7 GB)
        RX errors 0  dropped 0  overruns 0  frame 0
        TX packets 9253880  bytes 1479370777 (1.4 GB)
        TX errors 0  dropped 0 overruns 0  carrier 0  collisions 0

he-ipv6: flags=209<UP,POINTOPOINT,RUNNING,NOARP>  mtu 1480
        inet6 fe80::335b:79b6  prefixlen 64  scopeid 0x20<link>
        inet6 2001:470:1f20:1d2::2  prefixlen 64  scopeid 0x0<global>
        sit  txqueuelen 1000  (IPv6-in-IPv4)
        RX packets 782095  bytes 447475094 (447.4 MB)
        RX errors 0  dropped 0  overruns 0  frame 0
        TX packets 71534  bytes 6022694 (6.0 MB)
        TX errors 5  dropped 0 overruns 0  carrier 5  collisions 0

lo: flags=73<UP,LOOPBACK,RUNNING>  mtu 65536
        inet 127.0.0.1  netmask 255.0.0.0
        inet6 ::1  prefixlen 128  scopeid 0x10<host>
        loop  txqueuelen 1000  (Local Loopback)
        RX packets 7212  bytes 657864 (657.8 KB)
        RX errors 0  dropped 0  overruns 0  frame 0
        TX packets 7212  bytes 657864 (657.8 KB)
        TX errors 0  dropped 0 overruns 0  carrier 0  collisions 0
\end{termc}

Note that, in this example, the machine is multi-homed: the \texttt{ens3} interface is configured with the regular IPv6 address and the \texttt{he-ipv6} interface is configured with the Hurricane Electric IPv6 address. 

\section{Building our Test Network}\label{chap-TestNetTopo}

We want to reproduce a typical LPWAN network topology: an Internet-based application server communicates with IoT devices via an LPWAN Network Server (see \rfc{8376}).
We have already configured our SCHC instance on the "Core" side. On a second machine that will represent the IoT device, let's now install openSCHC as well.

~

The topology described in Figure~\ref{fig-topo} will be used throughout the rest of the tutorial. The compressed and fragmented SCHC packets will be tunneled over UDP across the Internet between the Device SCHC instance and the Core SCHC instance, as if they were carried over an LPWAN network.

 \begin{figure}[!ht] 
\centering 


	\begin{tikzpicture}[scale=0.8, transform shape]

	\draw (0,0) node(SCHC)[rectangle, minimum size=2cm, draw, fill=palerod]{}; 
	\draw(SCHC) node {\small{SCHC}};
	\draw(SCHC.north) node [above]{\small{Core}};
	
	\draw(SCHC.south) -- ++(0, -1) coordinate(a);
	\draw (a) -- ++(-2, 0);
	\draw (a) -- ++(2, 0);
	
	\path (SCHC) -- ++(5, 3) coordinate (HE);
	
	\draw (HE) node(HEBox) [cloud, minimum width=3cm, minimum height= 2cm, draw, fill=pink!20] {};
	\draw (HE) node  {\small{HE}};

	\draw [double, dashed] (HEBox.south) to [bend left=45] node [above, sloped] {IPv6} node [below, sloped] {over IPv4 tunnel} (SCHC.east);

	\path (HE) -- ++(5, 0) coordinate (App);

	\draw (App) node(AppBox) [cloud, minimum width=3cm, minimum height= 2cm, draw, fill=pink!20] {};
	\draw (App) node  {\small{Application}};
	\draw [thick] (HEBox.east) to node [above] {IPv6} (AppBox.west);

	\path (SCHC) -- ++(-6, 1) coordinate (dev);
	\draw (dev) node(SCHC_dev)[rectangle, minimum size=2cm, draw, fill=gray!10]{}; 
	\draw(dev) node {\small{SCHC}};
	\draw (dev) node[cloud, minimum size=4cm, draw]{}; 
	\draw (SCHC_dev.north) node [above] {\small{Device}};
	
	
	\draw [very thick] (SCHC_dev.south) to [bend right=80] node [below, sloped] {UDP}  node [above, sloped] {SCHC packet} (SCHC.south);
	

	\path (SCHC) -- ++(-6, -4) coordinate (dev);
	\draw (dev) node(SCHC_dev)[rectangle, minimum size=2cm, draw, dotted ]{}; 
	\draw(dev) node {\small{SCHC}};
	\draw (dev) node[cloud, minimum size=4cm, draw, dotted]{}; 
	\draw (SCHC_dev.north) node [above] {\small{Device}};
	
	
	\draw [very thick, dotted] (SCHC_dev.south) to [bend right=80]  (SCHC.south);
	

	\end{tikzpicture}
\caption{Test Network Topology} 
\label{fig-topo} 
\end{figure} 

The "Core" SCHC instance will act as a router between the IPv6 network and the constrained network, illustrated in our example by the UDP tunnel (but it could be made to be a real LPWAN network). 
~~

As per \rfc{8724}, the messages exchanged between SCHC entities are called "\Index{SCHC Packet}s". 

~~

Likewise, the LPWAN traffic from the device to the core is called "\Index{uplink}",

and the reverse traffic is called "\Index{downlink}" .

~~

One or several \Index{application}s will communicate with the devices.